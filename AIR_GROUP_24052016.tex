\documentclass[11pt]{article}

\usepackage[utf8]{inputenc}
\usepackage{graphicx}
\usepackage{hyperref}
\title{\textbf{Artificial Intelligence for Robotics\\ - Homework 8 -}}
\author{Kiran Vasudev, Patrick Nagel}
\date{Due date: 30.05.2016}
\begin{document}

\maketitle

\newpage
\tableofcontents

\newpage

\section{Answers to Questions}
	\subsection{Solve the Travelling Salesman Problem using Simulated Annealing:}
	\begin{itemize}
		\item{Get the best solution for the following computation times: 1, 3, 5, 10, 15, 30 minutes.}
		\item{Compare the performance between Random-restart Hill Climbing and Simulated Annealing.}
	\end{itemize}
\newpage
\subsection{Answer the following questions:}
	\begin{itemize}
		\item\textbf{What are constraint satisfaction problems?}\\
		%Patrick%
		
		\item\textbf{What are the components of a constraint satisfaction problem?}\\
		The components of a constraint satisfaction problem are Variables, Domains and constraints. \\\\
		\textbf X is a a set of Variables $$\left\{X_1,.....,X_n\right\}$$\\
		\textbf D is a a set of Domains $$\left\{D_1,.....,D_n\right\},$$ one domain for each variable\\\\
		\textbf C is a set of Constraints that allow a specific combination of values\\\\
		Each domain has an allowable set of values of each Variable $$\left\{v_1,..v_k\right\}$$ for each variable $X_i$.\\
		Each constraint consists of a pair $\langle scope, relation \rangle$ where scope is a tuple of variables that participate in the relation and the relation is the constraint that has to be applied to this tuple.
		
		\item\textbf{What is a solution for a constraint satisfaction problem?}\\
		%Patrick%
		
		\item\textbf{What is a constraint and how is it represented?}\\
		Each constraint consists of a pair $\langle scope, relation \rangle$ where scope is a tuple of variables that participate in the relation and the relation is the constraint that has to be applied to this tuple.\\ A constraint can be represented in a way where all the variables in the tuple satisfy the given constraint or a relation that supports two operations, namely, testing if a tuple is a member of a relation and enumerating the member of the tuple with the given relation.
		
		
		\item\textbf{What is a constraint graph?}\\
		%Patrick%
		
		\item\textbf{What are discrete variables and how are they divided?}\\
		A discrete variable is one who's domain is finite or countable infinite.\\ An example of a finite discrete variable is \textbf{Boolean}. An example of an infinite domain is a set of integers or strings.
		
		\item\textbf{What is preference?}\\
		%Patrick%
		
		\item\textbf{What is a successor function?}\\
		A successor function is that describes all possible actions that can convert the current state to another state. This function is used to move between different states. It is defines a relation of accessability among various states. 
		
		\item\textbf{What is Backtracking search?}\\
		%Patrick%
	\end{itemize}

\newpage
\subsection{In class you had a look at four methods for improving the efficiency of
the Backtracking search. Explain each approach in your own words.}
%K%

\newpage
\subsection{Represent the Sudoku puzzle as a constraint satisfaction problem. You
must include the domains, all variables and all constraints needed to
solve the puzzle.}
%Patrick%
\end{document}
